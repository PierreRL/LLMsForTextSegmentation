\section{Prompting Strategy}\label{Prompting Strategy}

The prompting strategy used in this work is a simple schema that is designed to be general and applicable to any LLM. The schema is as follows:

\begin{enumerate}
    \item The LLM is prompted with the input text, with integers in square brackets delimiting the sentence boundaries, few-shot examples of the task,a short instruction and a system prompt.
    \item Segments are validated. This means they must be not too long nor too short, and that they do not contain too many punctuation marks as a proportion of the segment length.
    \item Segments that are too long are recursively split into smaller segments through similar prompting strategy, but this time the LLM is asked to return a single segment boundary index.
    \item This process is repeated until all segments are short enough.
    \item Segments that are too short are merged with a neighbouring segment based on the semantic similarity to neighbouring sentences. This part could also be done via prompting, but we found this unnecessary.
\end{enumerate}

An example prompt is shown below. Note that this is not the exact prompt used in the experiments, but a simplified version for illustrative purposes.

\textbf{System:}

You are an expert linguist and a master of nuance in meaning of written text. You obey instructions. You do not hallucinate. You are not a chatbot. You are not a summariser.

\textbf{Prompt:} 

You are given a document with sentence boundaries marked by square brackets. Your task is to segment the document into coherent parts. Return a list of indices corresponding to the segment boundaries of the document. This list should ONLY be a list of integers, for example '1, 3, 5'. Some examples are shown below.

Text: 

It was a sunny day in the park. [1] The birds were singing. [2] The children were playing. [3] The adults were chatting. [4] The dogs were barking. [5] The sun was shining. [6] The day was perfect. [7] However, then the rain came. [8] The children ran for cover. [9] The adults laughed. [10] The dogs howled. [11] The sun disappeared. [12] The day was ruined. [13] Fortunately, the next day was sunny again. [14] But it was actually too hot! [15] The children were sweating. [16] The adults were fanning themselves. [17] The dogs were panting. [18] The sun was scorching. [19] The day was unbearable.

Segments: 

7, 13

\ldots \emph{[more examples]} \ldots

Text: 

The cat sat on the mat. [1] The dog sat on the floor. [2] The cat was black. [3] The dog was brown. [4] The cat was fluffy. [5] The dog was short-haired. [6] The cat was purring. [7] The dog was wagging its tail. [8] The cat was happy. [9] The dog was happy. [10] Then the cat went to London. [11] The dog went to Paris. [12] The cat saw the sights. [13] The dog saw the sights. [14] The cat ate fish and chips. [15] The dog ate croissants. [16] The cat drank tea. [17] The dog drank coffee. [18] The cat was happy. [19] The dog was happy.

Segments:

\textbf{End Prompt}

We use a similar prompt for the recursive prompting mechanism with the same system prompt. For example:

\textbf{Prompt:}

You are given a document with sentence boundaries marked by square brackets. Your task is to choose one segment boundary to split the document into two coherent parts. Return a single integer corresponding to the index of the segment boundary. This integer should be between 1 and the number of sentences in the document. Some examples are shown below.

Text:

The cat sat on the mat. [1] The cat was black. [2] The cat was fluffy. [3] The cat was purring. [4] The cat was happy. [5] On the other hand, the dog sat on the floor. [6] The dog was brown. [7] The dog was short-haired. [8] The dog was wagging its tail. [9] The dog was happy. [10]

Segment:

5

\ldots \emph{[more examples]} \ldots

Text: 

Jack and Jill went up the hill. [1] Jack fell down and broke his crown. [2] Jill came tumbling after. [3] This is a well known nursery rhyme that has been passed down through the generations. [4] It is a classic. [5] It is a favourite of many. [6] It is a favourite of mine. [7] It is a favourite of yours. [8] It is a favourite of everyone.

Segment:

3

\textbf{End Prompt}

These examples are not illustrative of the length or style of segmentations in our dataset, they merely serve to exemplify the prompting schema. The actual prompts used in the experiments were much longer and more complex, and included more examples which were more realistic. The system prompt was also more detailed and included more examples of what the model should not do, such as not repeating the same segment boundary multiple times, not exceeding the length of the input sentences and not getting stuck in a pattern of regular segment boundaries.