% 1. Introduction
% 1.1 Topic Segmentation Task Definition

\subsection{Topic Segmentation Task Definition}
Topic segmentation is the problem of dividing a string of text into constituent parts or ‘segments’. Each segment should be semantically self-contained such that it is about one thing. The precise definition of a segment should be dependent upon the specific use cases. For this work, segment boundaries shall always lie on sentence boundaries.

We can interpret segmentation as a binary classification task.  Given a list of input sentences $S$, of length $n$, the model must decide whether there exists a segment boundary between each pair of adjacent sentences. There are $n-1$ possible boundaries, and therefore our solution space is $2^{n-1}$.  Formally, the model must find a mapping $f$ from the list of sentences to a binary vector of length $n-1$:
\(
     f(S) = \textbf{y} 
\)
\( \textnormal{ where } \textbf{y}=\{y_1,y_2\ldots,y_{n+1}\} \textnormal{ and each of the } y_i\in\{0,1\}.
\)

Relative to generative tasks such as summarisation, the space of possible solutions is much smaller, but the problem remains subjective as where a boundary should lie can be ambiguous. Frequently, humans cannot agree on a correct solution \cite{TextTiling}.



 
