\section*{Limitations}

This work is comparable to a part of the work contained in~\cite{XingThesis}. However, this work is currently in the form of a PhD thesis which was made public in 2024. The experiments in this paper were conducted in 2023. This is why none of our experiments directly compare our method to their prompting methods or loss-based approaches. Our primary goal was to compare to the previously existing approach at Adarga, and to compare with other approaches available at the time. The code and datasets are no longer accessible to the authors due to their proprietary nature so we cannot rerun experiments on the same data with new prompting methods. We hope that future work can directly compare our prompting method with those proposed in~\cite{XingThesis} or loss-based approaches on larger datasets.

Results in this report are also subject to the subjective definition of a 'segment' as implied by manual segmentation, wikipedia heading placement or examples in the few-shot prompt. The conclusions in this paper may be subject to this implicit definition of a segment, but we are optimistic that methods and results presented here are equally valid with more flexible definitions of segments and across different.

This work is also limited by the size of datasets used. The numbers reported are from a subset of \emph{Wiki}. This was due to the computational resources available to us that were required to train and test the models. However, we tested on the full datasets with models for which it was computationaly and monetarily feasible and found that the subset was a large enough sample to be a good approximation of performance on the full dataset. We hope that future work can be conducted on larger datasets, both for fine-tuning and testing.