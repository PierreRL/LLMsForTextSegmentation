\section*{Limitations}

This work is comparable to a part of the work contained in~\cite{XingThesis}. However, this work is currently in pre-print as a PhD thesis which was made public in 2024. The research in this report is based on work from 2023, prior to the work from~\cite{XingThesis}. This is why none of our experiments do not cmopare to their methods. Our primary goal was to compare to the previously existing approach at Adarga, and to compare with other approaches available at the time. The code and datasets are no longer accessible to the authors to their proprietary nature. We hope that future work can directly compare our prompting method with those proposed in~\cite{XingThesis} or loss-based approaches on larger datasets.

Results in this report are also subject to the subjective definition in a 'segment' as encoded by manual segmentation or by wikipedia heading placement. This definiition However, we hope that methods and results presented here are valid with more flexible definitions of segments.

This work is also limited by the size of the dataset used. The dataset used in this work is a subset of the full Wikipedia dataset. This is due to the computational resources required to train the models. We hope that future work can be conducted on the full dataset to see if the results are consistent.