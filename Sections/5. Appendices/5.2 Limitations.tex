\section*{Limitations}

This work is comparable to a part of the work contained in~\cite{XingThesis}. However, this work is currently in pre-print as a PhD thesis which was made public in 2024. The research in this report is based on work from 2023, prior to the work from~\cite{XingThesis}. This is why none of our experiments directly compare our method to their prompting methods or loss-based approaches. Our primary goal was to compare to the previously existing approach at Adarga, and to compare with other approaches available at the time. The code and datasets are no longer accessible to the authors to their proprietary nature and so we cannot rerun experiments on the same data with new prompting methods. We hope that future work can directly compare our prompting method with those proposed in~\cite{XingThesis} or loss-based approaches on larger datasets.

Results in this report are also subject to the subjective definition in a 'segment' as encoded by manual segmentation or by wikipedia heading placement. The conclusions in this paper may be subject to this implicit definition of a segment, but we are optimistic that methods and results presented here are valid with more flexible definitions of segments and across different genres of text.

This work is also limited by the size of the dataset used. The dataset used in this work is a relatively small subset of the full Wikipedia dataset. This is due to the computational resources available to us that were required to train and test the models. We hope that future work can be conducted on larger datasets, both for fine-tuning and testing.