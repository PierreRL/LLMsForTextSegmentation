% 1. Introduction
% 1.1 Topic Segmentation Task Definition

% \subsection{Task Definition}

% We can interpret segmentation as a binary classification task.  Given a list of input sentences $S$, of length $n$, the model must decide whether there exists a segment boundary between each pair of adjacent sentences. There are $n-1$ possible boundaries, and therefore the solution space is $2^{n-1}$.  Formally, the model must find a mapping $f$ from the list of sentences to a binary vector of length $n-1$:
% \(
%      f(S) = \textbf{y} 
% \)
% \( \textnormal{ where } \textbf{y}=\{y_1,y_2\ldots,y_{n+1}\} \textnormal{ and each of the } y_i\in\{0,1\}.
% \)

% Relative to generative tasks such as summarisation, the space of possible solutions is much smaller, but the problem remains subjective as where a boundary should lie can be ambiguous. Frequently, humans cannot agree on a correct solution \cite{TextTiling}.

% Topic segmentation is the problem of dividing a string of text into constituent `segments'. Each segment should be semantically self-contained such that it is about one thing. The ambiguity in this definition leads to disagreements between humans annotators of `correct' segment boundaries~\citep{TextTiling}. For this work, segment boundaries always lie on sentence boundaries. We can then interpret segmentation as a binary classification task; given a list of input sentences, the model must decide whether there exists a boundary between each pair of adjacent sentences. While the solution space is smaller than generative tasks, the problem is also inherently subjective.



 
