% 4. Conclusion

Our work empirically compares generative LLMs with previous methods which use BERT embeddings and cosine similarity. In order to be more token-efficient and to provide guarantees that the original document will be unedited, we propose a new overlapping prompt schema, which centres on asking the LLM to return a list of indices corresponding to segment boundaries. We also support the use of boundary similarity and its associated information recall metrics as an evaluation for topic segmentation. Results indicate that LLMs can be more effective segmenters than existing methods where more nuanced segmentations are required, but that when the input is noisy or the segment boundaries are clear, BERT-based methods are more reliable. Future work should focus on addressing highlighted issues with LLMs, such as the regular patterns found in segmentations, and a more thorough comparison of promptin methods. Lastly, larger human-annotated datasets should be constructed and used to better assess generalisation capabilities.