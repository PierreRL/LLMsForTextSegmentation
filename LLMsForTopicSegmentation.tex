\documentclass{article}

\usepackage{amsmath} 
\usepackage{microtype}
\usepackage{graphicx}
\usepackage{subfigure}
\usepackage{booktabs}
\usepackage{hyperref}
\usepackage{authblk}

\usepackage[accepted]{icml2023}

\icmltitlerunning{Submission and Formatting Instructions for ICML 2021}

\begin{document}

\twocolumn[


\icmltitle{Topic Segmentation Using Generative Language Models}

\begin{icmlauthorlist}
\icmlauthor{Pierre Mackenzie}{Adarga}
\icmlauthor{Maya Shah}{Adarga}
\end{icmlauthorlist}
\centering
\textsuperscript{1}Adarga, London\\
\vskip 0.3in
]

\begin{abstract}
\emph{
    Topic segmentation using generative Large Language Models (LLMs) remains relatively unexplored. Previous methods use lexical or semantic similarity between parts of a document to decide on boundaries, but they lack the long range dependency and vast knowledge contained in LLMs. Here, we propose a new prompting strategy and compare to semantic similarity-based methods. Results show that LLMs can be more effective segmenters than existing methods, but issues remain to be solved before they can be relied upon for topic segmentation.
}
\end{abstract}

\section{Introduction}

% 1
% ---------- Introduction ----------
\subsection{The Topic Segmentation Problem}
% 1. Introduction
% 1.1 Topic Segmentation Task Definition

\subsection{Topic Segmentation Task Definition}
Topic segmentation is the problem of dividing a string of text into constituent parts or ‘segments’. Each segment should be semantically self-contained such that it is about one thing. The precise definition of a segment should be dependent upon the specific use cases. For this work, segment boundaries shall always lie on sentence boundaries.

We can interpret segmentation as a binary classification task.  Given a list of input sentences $S$, of length $n$, the model must decide whether there exists a segment boundary between each pair of adjacent sentences. There are $n-1$ possible boundaries, and therefore our solution space is $2^{n-1}$.  Formally, the model must find a mapping $f$ from the list of sentences to a binary vector of length $n-1$:
\(
     f(S) = \textbf{y} 
\)
\( \textnormal{ where } \textbf{y}=\{y_1,y_2\ldots,y_{n+1}\} \textnormal{ and each of the } y_i\in\{0,1\}.
\)

Relative to generative tasks such as summarisation, the space of possible solutions is much smaller, but the problem remains subjective as where a boundary should lie can be ambiguous. Frequently, humans cannot agree on a correct solution \cite{TextTiling}.



 

\subsection{Related Work}

Previous methods of topic segmentation use either lexical or semantic similarity, and a variety of different machine learning approaches. However, there has been no research into the usage of LLMs for topic segmentation. LLMs have been shown to be effective at a variety of NLP tasks due to their vast general knowledge of language, a skill which is also required for topic segmentation.


% 2
% ---------- Method ----------
\section{Method}
% 2. Method
% 2.1 Datasets
\subsection{Datasets}

There are 4 types of datasets used in the experiments in this work: a small human-annotated dataset, a scrape of English wikipedia, a 'concatenated' wikipedia scrape and a synthetic GPT3.5 generated dataset.

\subsubsection{Human-Annotated Dataset}

Lacking the resources to create a large dataset annotated by humans, this work uses a very small manually segmented dataset of 10 documents. These documents are a mix of news articles, wikipedia articles, and miscellaneous documents such as podcast transcripts and scientific reports. This was intended to represent varying difficulties of segmentation, and provide examples which could be manually inspected to interpret segmenter behaviour. Only one annotator was used, and the segments were created by hand. Further work might involve multiple annotators on a much larger dataset, with a more rigorous process to ensure consistency and quality.

\subsubsection{Wikipedia Dataset}

A plain text English wikipedia scrape\footnote{\url{https://www.kaggle.com/datasets/ltcmdrdata/plain-text-wikipedia-202011/data}} was used articles where headings were delimited by special characters. The articles were then automatically segmented based on headings, and filtered to remove articles with very few segments, too short segments, or too much punctuation such as tables and figures. After this process, there remained approximately 1000 segmented articles. We generate two version of this dataset: one with headings removed, and one with headings included. This was used during evaluation to investigate if a model is merely segmenting based on headings.

\subsubsection{Concatenated Wikipedia Dataset}

We randomly sampled segments from the previous Wikipedia dataset and concatenated them in order to form new incoherent articles, with segments drawn from completely different domains. Intuitively, this dataset should be easier to segment as there are no semantic links between segments.

\subsubsection{Synthetic Dataset}

The final dataset used was generated synthetically by GPT-3.5. The source data was a mix of CTC sentinel data\footnote{\url{https://ctc.westpoint.edu/ctc-sentinel/}}, UN-Peacekeeping corpus\footnote{\url{https://peacekeeping.un.org/en/reports}} and an internal dataset at Adarga. These documents were segmented by querying OpenAI's API with the model \texttt{gpt-3.5-turbo-16k} using the overlapping prompt schema defined in section~\ref{LLM-Based Text Segmentation}. This dataset was exclusively used for fine-tuning the \emph{FlanT5}\ref{FlanT5} model. It was not used for evaluation as the results would be biased in favor of the generative models.
% 2. Method
% 2.2 Evaluation
\subsection{Evaluation}\label{evaluation}

We follow the work in~\cite{fournier-2013-B} which proposes the Boundary Similarity metric and associated precision/recall. The metric pairs segment boundaries between a hypothesised and references segmentation. Exact matches score 1 and no match scores 0, whilst matches within a distance $n$ score linearly in the distance. Boundary Similarity (B) is the mean score, while Boundary Precision/Recall (BP/BR) are the mean score of matched hypothesis/reference boundaries, respectively. For further justification for the use of boundary similarity as opposed to more traditional metrics such as WD and Pk~\cite{HearstW2002}, see~\cite{fournier-2013-B}, or our own investigations\footnote{\url{https://github.com/PierreRL/segmenter-evaluation-metrics}}.

% or our own investigations at \href{https://github.com/PierreRL/segmenter-evaluation-metrics}{segmenter evaluation metrics}.
% 2. Method

\subsection{LLM-Based Text Segmentation}\label{LLM-Based Text Segmentation}

% 2.3 LLM-Based Text Segmentation

How can we get LLMs to output segment boundaries?

We might first consider passing in the input text as a prompt and asking the LLM to copy out the text, adding markers indicating where it has placed boundaries as is suggested by~\citep{XingThesis}. However, not only is this wasteful of tokens, especially if the input text is long, but crucially, the GLM may fail to copy the input accurately, may change the formatting, and we found through qualitative experimentation that boundary placement was not any better than our final method. These problems are addressed by~\citep{XingThesis} through repeated prompting until the input and output sequence lengths match, but this still does not guarantee integrity of the data. In our use case, guaranteeing that the input data would remain the same was of the utmost importance, so we used a different strategy.

\subsubsection{Prompting Method}

We first annotate the text with indices between each sentence. Here is an example: `Hello World. [1] The sky is blue. [2] The sun is is yellow. [3] The grass is green. [4] Machine learning is a rapidly evolving field'. We then ask the LLM to return a list of indices corresponding to boundaries. In the previous example, the ideal response might be `1,4'. In practice, the texts and list of segment boundaries are much longer.

We add a system prompt which describes the segmentation task, desired output format and primes the model to be a talented linguist. We also add a variety of short examples in line with the few-shot prompting technique~\citep{FewShotLearners}, which improved performance. This did not necessarily reflect the fact that the LLM learned how to segment better. Instead, through manual testing, we suspect that it learned the ideal segment length and amount of information that should be contained within a segment, which was implicitly contained in the few-shot prompt (and also the testing datasets, therefore increasing performance). This suggests that different implicit definitions of a segment could be imposed by a few-shot prompt to a LLM, dependent upon use case.

An example prompt is contained in [XXX Appendix A].

\subsubsection{Overlapping Prompts}

This prompting method works so long as the input text is within the context window of the LLM. At the time of experimentation, and with the use of gpt-3.5-turbo, we had a limit of 16k tokens. Many of our input texts exceeded this limit. Therefore, we needed to split up these long documents into smaller chunks that can be processed by the LLM. However, this cannot be done by simply splitting every at the nearest sentence before every 16k new tokens for two reasons. Firstly, we do not know whether this sentence boundary should serve as a segment boundary, and secondly, the LLM loses valuable context which helps to choose where to place boundaries at the extremes of the 16k tokens.

Therefore, we instead send 16k prompts with some overlap. The overlap is calculated as twice the maximum segment length. In our experiments, we set a maximum segment length of 750 tokens, thus there is an overlap of 1500 tokens between prompts. We must then decide which boundaries to accept in this overlapping region. Given two generations which were prompted by 1500 overlapping tokens, we choose to accept the segment boundaries contained within the first 750 tokens of the overlapping section from the first generation, and the boundaries in the final 750 tokens from the second prompt. We did not experiment with involving the responses from both outputs, but found that there seemed to be no sign of degrading performance towards segment boundaries.

While this method is wasteful of up to 1500 tokens per prompt, this is a small enough fraction of the 16k context that we were satisfied with the solution. If maximum segment lengths were much longer, say 5k tokens, the context may need to be limited more severely.

\subsubsection{Segment Validation}

We also performed some validation on the segments returned by the LLM. This primarily involved verifying that the returned segments are within a maximum and minimum segment length. Segments that are too short (for example, a model would sometimes return just a heading), were concatenated with another segment, and segments that are too long were recursively segmented by the same model, but with another prompt that asks the model to generate only one boundary at a time, which uses a similar few-shot prompting strategy to above. This way, a segment that is too long will be split in 2 recursively until all segments are within the specified lengths.

% 3. Experiments
\section{Experiments}\label{Results}
% 3.1 Results
\subsection{Models}

\subsubsection{Baselines}

We use two naive baselines as a point of reference. First, a segmenter which splits every n sentences. We decided to split every 5 sentences which we call the \emph{Split5Segmenter}. We also define a \emph{RandomF0.1Segmenter} which splits at 10\% of boundaries, placed randomly, with each potential boundary equally likely.

\subsubsection{BERT Segmenter}

Our existing method generates a sequence of sentence similarities using sentence embeddings generated by \cite{SentenceBERT}. Similarities are calculated as a weighted sum of the cosine similarity to the previous $n$ sentences. Ideal boundaries are then generated as troughs in the sequence of similarities, before further processing to ensure there are no segments that are too long or too short, either in sentence length or in token length. We name this the \emph{BERTSegmenter}. Further details are omitted for proprietary reasons.

\subsubsection{BERT-Graph Segmenter}

\cite{MasimilianoSegmenter} describes 'text tiling' (topic segmentation) using BERT-generated similarity scores followed by graph clustering to find the best segments. This follows a similar methodology as the previous Exact details of this method can be found in the linked article. This model is called \emph{BERTGraphSegmenter} in our experiments. Code for this model was copied from the repository linked in the article.

\subsubsection{GPT-3.5}

OpenAI's \texttt{gpt-3.5-turbo-16k} was queried using the prompting and segment validation strategy defined in Section 2. We call this model \emph{GPT3.5}. We used the largest model available to us, which is the 16k token context window.

\subsubsection{Flan-T5-Finetuned}\label{FlanT5}

Took Flan-T5 large~\citep{FlanT5} and fine-tuned it a combination of wiki, concatenated wiki and synthetic data.
% 3.2 Results

\subsection{Quantitative Results}

Due to resource constraints, we could not test \emph{GPT3.5} on the full wikipedia or full concatenated wikipedia datasets. Instead, we took the largest subset that fit within resource constraints. We evaluated all other models on the full datasets to verify that similar results are obtained. Further details on the experimental procedure are witheld for proprietary reasons.

We evaluate primarily using the previously discussed boundary similarity metric with $n=5$. We also looked at precision and recall which helped us characterize the behavior of each model. Results can be found in Table~\ref{tab:quant_results}. 

We find that ...

Results were also taken with $n=2$ and $n=10$. Full results for both the smaller and larger datasets and all evalutation metrics can be found in the Appendix~\ref{Appendix}.

\begin{table}[ht]
\centering
\begin{tabular}{c|ccc}
& Human & Wiki & Wiki-concat \\ \hline
GPT3.5  & Row 1 Data & Row 1 Data & Row 1 Data \\ 
Flan-T5  & Row 2 Data & Row 2 Data & Row 2 Data \\ 
BERTGraph & Row 3 Data & Row 3 Data & Row 3 Data \\ 
BERT & Row 4 Data & Row 4 Data & Row 4 Data \\
RandomF0.1& Row 5 Data & Row 5 Data & Row 5 Data \\
Split5 & Row 6 Data & Row 6 Data & Row 6 Data \\
\end{tabular}
\caption{Boundary similarity with $n=5$ for the different models on the human-annotated, Wikipedia, and concatenated Wikipedia datasets.}
\label{tab:quant_results}
\end{table}

\subsection{Qualitative Results}

Through manual testing with the human-annotated dataset, we found that \emph{GPT3.5} generally found the boundaries which seemed most reasonable from a human perspective, especially for simple documents. The 


BERT segmenters would find reasonable segments, but after the manual gluing and splitting procedure, would often lead to off-by-1 errors.

However, for documents with far more complex documents such as a podcast transcript, or with messy data like tables or artefacts from pdf to text conversions, \emph{GPT3.5} would sometimes return indices with a regular pattern. For example, the GLM might return '$[1,15,22, ..., 76, 79, 82, 85, 88, ..., 184, 187, ...]$'. Often, the pattern would continue far beyond the number of sentences in the input indicating that the model became stuck in a regular pattern. Perhaps better prompt engineering, a more rigorous data-processing procedure or the use of newer models would help, but our current approach was resource constrained and required the ability to pass noisy documents to the model. A more thorough investigation of the logits computed by the model is required to understand how and when this occurs, and how to mitigate it.



% 4. Conclusion
\section{Conclusion}\label{Conclusion}
% 4. Conclusion

Our work empirically compares generative LLMs with methods which use BERT embeddings and cosine similarity for topic segmentations. We propose a new overlapping prompting method that is token-efficient and provides guarantees of the integrity of the data passed into the model. We also support the use of boundary similarity and its associated information recall metrics for evaluation. Results indicate that LLMs can be more effective segmenters where more nuanced segmentations are required. On the contrary, when the input is noisy or the segment boundaries are clear, BERT-based methods may be more reliable. Future work should involve a thorough comparison of different prompting methods and addressing highlighted issues with LLM outputs using our method. Lastly, larger human-annotated datasets should be constructed to better assess generalisation capabilities.

% Bibliography
\bibliography{references.bib}
\bibliographystyle{cs502}


% Appendix

\appendix
\section{Appendix}\label{Appendix}
\section{Prompting Strategy}\label{Prompting Strategy}

The prompting strategy used in this work is a simple schema that is designed to be general and applicable to any LLM. The schema is as follows:

\begin{enumerate}
    \item The LLM is prompted with the input text, with integers in square brackets delimiting the sentence boundaries, few-shot examples of the task,a short instruction and a system prompt.
    \item Segments are validated. This means they must be not too long nor too short, and that they do not contain too many punctuation marks as a proportion of the segment length.
    \item Segments that are too long are recursively split into smaller segments through similar prompting strategy, but this time the LLM is asked to return a single segment boundary index.
    \item This process is repeated until all segments are short enough.
    \item Segments that are too short are merged with a neighbouring segment based on the semantic similarity to neighbouring sentences. This part could also be done via prompting, but we found this unnecessary.
\end{enumerate}

An example prompt is shown below. Note that this is not the exact prompt used in the experiments, but a simplified version for illustrative purposes.

\textbf{System:}

You are an expert linguist and a master of nuance in meaning of written text. You obey instructions. You do not hallucinate. You are not a chatbot. You are not a summariser.

\textbf{Prompt:} 

You are given a document with sentence boundaries marked by square brackets. Your task is to segment the document into coherent parts. Return a list of indices corresponding to the segment boundaries of the document. This list should ONLY be a list of integers, for example '1, 3, 5'. Some examples are shown below.

Text: 

It was a sunny day in the park. [1] The birds were singing. [2] The children were playing. [3] The adults were chatting. [4] The dogs were barking. [5] The sun was shining. [6] The day was perfect. [7] However, then the rain came. [8] The children ran for cover. [9] The adults laughed. [10] The dogs howled. [11] The sun disappeared. [12] The day was ruined. [13] Fortunately, the next day was sunny again. [14] But it was actually too hot! [15] The children were sweating. [16] The adults were fanning themselves. [17] The dogs were panting. [18] The sun was scorching. [19] The day was unbearable.

Segments: 

7, 13

\ldots \emph{[more examples]} \ldots

Text: 

The cat sat on the mat. [1] The dog sat on the floor. [2] The cat was black. [3] The dog was brown. [4] The cat was fluffy. [5] The dog was short-haired. [6] The cat was purring. [7] The dog was wagging its tail. [8] The cat was happy. [9] The dog was happy. [10] Then the cat went to London. [11] The dog went to Paris. [12] The cat saw the sights. [13] The dog saw the sights. [14] The cat ate fish and chips. [15] The dog ate croissants. [16] The cat drank tea. [17] The dog drank coffee. [18] The cat was happy. [19] The dog was happy.

Segments:

\textbf{End Prompt}

We use a similar prompt for the recursive prompting mechanism with the same system prompt. For example:

\textbf{Prompt:}

You are given a document with sentence boundaries marked by square brackets. Your task is to choose one segment boundary to split the document into two coherent parts. Return a single integer corresponding to the index of the segment boundary. This integer should be between 1 and the number of sentences in the document. Some examples are shown below.

Text:

The cat sat on the mat. [1] The cat was black. [2] The cat was fluffy. [3] The cat was purring. [4] The cat was happy. [5] On the other hand, the dog sat on the floor. [6] The dog was brown. [7] The dog was short-haired. [8] The dog was wagging its tail. [9] The dog was happy. [10]

Segment:

5

\ldots \emph{[more examples]} \ldots

Text: 

Jack and Jill went up the hill. [1] Jack fell down and broke his crown. [2] Jill came tumbling after. [3] This is a well known nursery rhyme that has been passed down through the generations. [4] It is a classic. [5] It is a favourite of many. [6] It is a favourite of mine. [7] It is a favourite of yours. [8] It is a favourite of everyone.

Segment:

3

\textbf{End Prompt}

These examples are not illustrative of the length or style of segmentations in our dataset, they merely serve to exemplify the prompting schema. The actual prompts used in the experiments were much longer and more complex, and included more examples which were more realistic. The system prompt was also more detailed and included more examples of what the model should not do, such as not repeating the same segment boundary multiple times, not exceeding the length of the input sentences and not getting stuck in a pattern of regular segment boundaries.

\section{Full Results}\label{Full Results}



\end{document}

